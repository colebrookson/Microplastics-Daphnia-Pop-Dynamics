% Options for packages loaded elsewhere
\PassOptionsToPackage{unicode}{hyperref}
\PassOptionsToPackage{hyphens}{url}
%
\documentclass[
]{article}
\usepackage{lmodern}
\usepackage{amssymb,amsmath}
\usepackage{ifxetex,ifluatex}
\ifnum 0\ifxetex 1\fi\ifluatex 1\fi=0 % if pdftex
  \usepackage[T1]{fontenc}
  \usepackage[utf8]{inputenc}
  \usepackage{textcomp} % provide euro and other symbols
\else % if luatex or xetex
  \usepackage{unicode-math}
  \defaultfontfeatures{Scale=MatchLowercase}
  \defaultfontfeatures[\rmfamily]{Ligatures=TeX,Scale=1}
\fi
% Use upquote if available, for straight quotes in verbatim environments
\IfFileExists{upquote.sty}{\usepackage{upquote}}{}
\IfFileExists{microtype.sty}{% use microtype if available
  \usepackage[]{microtype}
  \UseMicrotypeSet[protrusion]{basicmath} % disable protrusion for tt fonts
}{}
\makeatletter
\@ifundefined{KOMAClassName}{% if non-KOMA class
  \IfFileExists{parskip.sty}{%
    \usepackage{parskip}
  }{% else
    \setlength{\parindent}{0pt}
    \setlength{\parskip}{6pt plus 2pt minus 1pt}}
}{% if KOMA class
  \KOMAoptions{parskip=half}}
\makeatother
\usepackage{xcolor}
\IfFileExists{xurl.sty}{\usepackage{xurl}}{} % add URL line breaks if available
\IfFileExists{bookmark.sty}{\usepackage{bookmark}}{\usepackage{hyperref}}
\hypersetup{
  pdftitle={Analysis Plan},
  pdfauthor={Cole B. Brookson},
  hidelinks,
  pdfcreator={LaTeX via pandoc}}
\urlstyle{same} % disable monospaced font for URLs
\usepackage[margin=1in]{geometry}
\usepackage{graphicx,grffile}
\makeatletter
\def\maxwidth{\ifdim\Gin@nat@width>\linewidth\linewidth\else\Gin@nat@width\fi}
\def\maxheight{\ifdim\Gin@nat@height>\textheight\textheight\else\Gin@nat@height\fi}
\makeatother
% Scale images if necessary, so that they will not overflow the page
% margins by default, and it is still possible to overwrite the defaults
% using explicit options in \includegraphics[width, height, ...]{}
\setkeys{Gin}{width=\maxwidth,height=\maxheight,keepaspectratio}
% Set default figure placement to htbp
\makeatletter
\def\fps@figure{htbp}
\makeatother
\setlength{\emergencystretch}{3em} % prevent overfull lines
\providecommand{\tightlist}{%
  \setlength{\itemsep}{0pt}\setlength{\parskip}{0pt}}
\setcounter{secnumdepth}{-\maxdimen} % remove section numbering

\title{Analysis Plan}
\author{Cole B. Brookson}
\date{5/20/2020}

\begin{document}
\maketitle

\hypertarget{step-by-step-for-bayesian-debtox-model-fitting}{%
\subsection{Step by Step for Bayesian DEBtox Model
Fitting}\label{step-by-step-for-bayesian-debtox-model-fitting}}

\hypertarget{general-idea}{%
\subsubsection{General Idea}\label{general-idea}}

Whole idea is to calculate two things:

\begin{enumerate}
\def\labelenumi{\arabic{enumi}.}
\tightlist
\item
  The probability \(S(t,c)\) to be alive at time \(t\) and exposure
  concentration \(c\) given by: \[ S(t,c) = ((1-m)(1-s_A(c)))^t\] where
  \(m\) is the blank daily probability of death
\item
  Calculate the DEBtox reproduction functions \(R_F(t,c)\) as calibrated
  for \emph{D. magna}.
\end{enumerate}

From those two things, we then go to an age-based matrix model where for
each generation, daphnids move from age \emph{i} to \emph{i+1} according
to age and generation (F0, F1, F2, F3) specific survival rates,
\(P_{i,F}(c)\), calculated considering the survival function
\(S_{F}(t,c)\) given in 1. above, where:
\[P_{i,F}(c) = \frac{S_F(t(i+1),c)}{S_F(t(i),c)}\] In addition, age- and
generation-specific fecundty rates \(Fec_{i,F}(c)\) are calculated from
the reproduction functions \(R_F(t,c)\). Assumed is a birth pulse model
and a prebreeding census:
\[FEC_{i,F}(c) = \int_{i}^{i+1}P_{1,F}(c)R_F(t-3,c)dt \] Goal is to then
get \(\lambda\) from Leslie matrix. Then, explore cases where the
population goes extinct. Use some sort of approach (MC simulations?) to
account for model uncertainty at organism level. For each generation,
draw 1000 parameter from their joint posterior distribution and for each
parameter set, survival and fecundity rates are calculated over the rate
of concentrations and the corresponding \(\lambda\) and extinction
probability are derived.

\hypertarget{individual-level-modeling}{%
\subsubsection{Individual-level
Modeling}\label{individual-level-modeling}}

Goal of the individual-level model is to quantify effects of
microplastics on growth and reproduction via the DEBtox model. There are
5 models of the DEBtox framework, two direct effect models (hazard \&
costs) and three indirect models (growth, assimilation, maintenance).

\hypertarget{growth}{%
\paragraph{Growth:}\label{growth}}

\[ GROWTH: \frac{dl}{dt} = r\frac{f+g}{f + g(l+s)}(f-l)\]
\[ REPRODUCTION: R = \frac{R_M}{1-l^3_p} \left( fl^2\frac{g(1+s) + l}{g(1+s) + f}-l^3_p\right)\]

\hypertarget{assimilation}{%
\paragraph{Assimilation:}\label{assimilation}}

\[ \frac{dl}{dt} = r\frac{f+g}{f(1-s)+g}(f(1-s)-l)\]

\[ R = \frac{R_M}{1-l^3_p}\left(f(1-s)l^2\frac{g+ l}{g+g(1-s)}-l^3_p\right)\]

\hypertarget{maintenance}{%
\paragraph{Maintenance:}\label{maintenance}}

\[ \frac{dl}{dt} = r(f-l(1+s))\]

\[ R = \frac{R_M}{1-l^3_p}(1+s)\left(fl^2\frac{g(1+s)^{-1} + l}{g+ f}-l^3_p\right)\]

\hypertarget{costs}{%
\paragraph{Costs:}\label{costs}}

\[ \frac{dl}{dt} = r(f-l)\]
\[ R = \frac{R_M}{1-l^3_p}\left(fl^2\frac{g + l}{g + f}-l^3_p\right)(1+s)^{-1}\]

\hypertarget{hazard}{%
\paragraph{Hazard:}\label{hazard}}

\[ \frac{dl}{dt} = r(f-l)\]
\[ R = \frac{R_M}{1-l^3_p}\left(fl^2\frac{g + l}{g + f}-l^3_p\right)e^{-s}\]

Model end points are body length \(L\) (or the scaled body length
\(l = \frac{L}{L_m}\), where \(L_m\) is the control max body length) and
the reproduction rate \(R\) (number of offspring per mother per time
unit). These are functions of exposure time \(t\) and exposure
concentration \(c\). So there are nine parameters:

\begin{enumerate}
\def\labelenumi{\arabic{enumi}.}
\tightlist
\item
  \(L_m\) - max body length
\item
  \(l_p\) - scaled body length at puberty
\item
  \(r\) - von Bertalanffy growth rate
\item
  \(R_M\) - maximum reproduction rate
\item
  \(NEC\) - the No Effect Concentration
\item
  \(c_*\) - tolerance concentration (according to the effect model)
\item
  \(k_e\) - elimination rate
\item
  \(f\) - reference value for control organisms at optimal temperature -
  assumed to be 1
\item
  \(g\) - investment ratio - fixed at 1
\end{enumerate}

\(L_m, r, l_p, R_M\) are common to all organisms. In addition, the
theoretical scaled length \(l_{h,i}\), and reproduction rate \(R_{h,i}\)
are latent variables in this framework. The scaled concentration in the
organisms \(c_{q}\) is defined as \(\frac{dc_q}{dt} = k_e(c-c_q)\).

The stress induced by the toxicant (plastic in our case) is modeled as a
stress function \(s(c_q(t,c))\) where \(c_q\) is the internal
concentration scaled by a bioconcentration factor, so that \(c_q\) is
homogenous to an exposure concentration. The value of the stress
function is positive when the scaled concentration inside the organisms
(\(c_q\)), exceeds the NEC. This is modeled by
\[s(c_q) = c^{-1}_* (c_q - NEC)_+\] where
\((c_q-NEC)_+ = max(0, c_q - NEC)\). \(c_*\) is the tolerance
concentration, where \(* = A, G, M, R, or H\) depending on the effect
model considered.

Two important assumptions of direct effects on reproduction:

\begin{enumerate}
\def\labelenumi{\arabic{enumi}.}
\tightlist
\item
  there is an additional mortality during oogenesis (hazard model)
\item
  OR there is an increase in the energy costs per egg (cost model)
\end{enumerate}

Three assumptions of indirect effects on reproduction

\begin{verbatim}
1. There is additional growth costs (growth model)
2. There is reduced incoming inergy (assimilation model)
3. There is additional maintenance costs (maintenance model)
\end{verbatim}

Model end points are body length \(L\) or the scaled body length
\(l = L/L_m\) where \(L_m\) is the control max body length and the
reproduction rate \(R\) (\# of offspring per mother per time unit).
These are expressed as functions of the two covariates: exposure time
\(t\) and the exposure concentration \(c\).

There are 7 parameters of interest involved in the equations:

\begin{enumerate}
\def\labelenumi{\arabic{enumi}.}
\tightlist
\item
  max body length (Lm, milimetres)
\item
  scaled body length at puberty (lp, dimensionless)
\item
  von Bertalanffy growth rate (gamma, d\^{}-1)
\item
  max reproduction rate (Rm, d\^{}-1)
\item
  the No effect concentration (NEC, mass L\^{}-1)
\item
  tolerance concentration (c* = H, R, A, G, or M according to the effect
  model)
\item
  elimination rate (ke d-1)
\end{enumerate}

Lm, gamma, lp, and Rm are common to all organisms in an experiment and
do not depend on concentration Lm, and Rm are not always reached by the
organisms affected by a contaminant

End points body length and cumulative reproduction data supposedly
follow normal distributions for body length with a mean equal to the
theoretical body length and a standard deviation (sigma g) and the cum.
reprod with a mean equal to the cumulative number of offspring per
mother and a standard deviation sigma r Sigma g represents the invidual
variability of body length data with variability between beakers ignored
and sigma r represents the variability of reproduction between beakers.
Scaled length (lh,i) and reproduction rate, Rh,i, which are usually end
points in DEBtox models are latent variables in the Bayesian framework
and depend directly on parameters

\hypertarget{control-equations}{%
\subsubsection{Control Equations}\label{control-equations}}

For the control organisms: \[\frac{dl}{dt} = r(f-1)\]

\[ R = \frac{R_M}{l-l^3_{p}} * (fl^2 \frac{g+l}{g+f} - l^3_{p})\] The
stress induced by the toxicant is modeled as a stress function
\(s(c_q(t,c))\) where \(c_q\) is the internal concentration scaled by a
bioconcentration factor, so that \(c_q\) is homogeneous to an exposure
concentration. \(s(c_q(t,c))\) is positive when the scaled concentration
exceeds an exposure concentration called the No Effect Concentration
(NEC), and null otherwise. \[s(c_q) = c^{-1}_*(c_q - NEC)_+\] with
\((c_q-NEC)_+ = max(0, c_q-NEC)\). \(c_*\) is the tolerance
concentration. \(* = A, G, M, R, or H\) depending on the effect model
considered. The scaled concentration inside the organism (\(c_q\))
follows: \[\frac{dc_q}{dt} = k_e(c-c_q) \] where \(k_e\) is the
elimination rate and \(c\) is the exposure concentration.

\end{document}
